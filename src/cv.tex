%% start of file `template.tex'.
%% Copyright 2006-2013 Xavier Danaux (xdanaux@gmail.com).
%
% This work may be distributed and/or modified under the
% conditions of the LaTeX Project Public License version 1.3c,
% available at http://www.latex-project.org/lppl/.

\documentclass[11pt,a4paper,sans]{moderncv}
\moderncvstyle{casual}                            % style options are 'casual' (default), 'classic', 'oldstyle' and 'banking'
\moderncvcolor{blue}
% character encoding
\usepackage[utf8]{inputenc}

% adjust the page margins
\usepackage[scale=0.75]{geometry}
\usepackage{import}

\usepackage{todonotes}

\usepackage{xpatch}
\xpatchcmd\cventry{,}{}{}{}


\newcommand\Colorhref[3][cyan]{\href{#2}{\small\color{#1}#3}}

\renewcommand*{\namefont}{\fontsize{14}{18}\mdseries\upshape}
\renewcommand*{\titlefont}{\fontsize{14}{18}\mdseries\upshape}

\makeatletter
\renewcommand*{\bibliographyitemlabel}{\@biblabel{\arabic{enumiv}}}
\patchcmd{\thebibliography}
  {\setlength{\labelwidth}{\hintscolumnwidth}}
  {\setlength{\labelwidth}{0pt}}
  {}
  {}
\makeatother


% personal data
\name{Jos\'e Pablo}{Cambronero}
\email{josepablocam@gmail.com}
\homepage{www.josecambronero.com}
\phone{215-900-5308}
\extrainfo{\httplink{www.github.com/josepablocam}}



\usepackage{etoolbox}
\patchcmd{\makecvtitle}{{minipage}[b]}{{minipage}[t]}{}{}

\begin{document}
\makecvtitle

\section{Overview}
\small{
(As of \today{}) I'm a researcher working on using
machine learning for code to develop state-of-the-art
AI-assisted developer tools to make writing, fixing, and using
software easier and more enjoyable.
}

\section{Education}

\vspace{5pt}

\subsection{Academic Qualifications}

\vspace{5pt}

\cventry{2016-2021}{PhD in Computer Science\newline}{Massachusetts Institute of Technology}{Cambridge, MA}{}{}

\cventry{2013-2016}{Masters in Computer Science\newline}{New York University: Courant Institute of Mathematical Sciences}{NY, NY}{}{\textit{GPA: 3.89, MS Research/Thesis Fellowship Award Fall 2015, funding  work on A2Q (an order-aware optimizing query compiler for AQuery)}}

\cventry{2007-2011}{Bachelor of Arts in Economics and Minor in German Studies\newline}{University of Pennsylvania}{Philadelphia, PA}{}{\textit{GPA: 3.93, Phi Beta Kappa, Summa Cum Laude, Dean's List (08, 09, 10)}}


\section{Industry Work Experience}
% \subsection{Full time}
%
% \vspace{6pt}
%
%
%
%
% % \end{itemize}
%
% \subsection{Internships}
\vspace{6pt}
\cventry{07/2024 -- current}{Staff Software Engineer}{DevAI Team}{Google}{Atlanta, GA}{
}{}
\cventry{06/2022 -- 05/2024}{Senior Researcher}{PROSE Team}{Microsoft}{Remote}{
}{}
\cventry{06/2021 -- 06/2022}{Researcher}{PROSE Team}{Microsoft}{Remote}{
  \begin{itemize}
    \item Working on program synthesis technologies for a variety of developer, 
    data scientist, and end-user applications. 
    A lot of my work focuses on developing and applying large 
    language models to programming tasks, such as program repair and natural language to code synthesis.
    As part of my job, I also manage and mentor junior researchers through the PROSE
    research fellowship program.
  \end{itemize}
}{}

\cventry{Summer 2020}{Intern}{Facebook AI Research}{Facebook}{Remote}{
  \begin{itemize}
    \item Worked with the SysML team at FAIR on a novel tensor compiler,
    writing C++ for JIT compilation, benchmarking against Halide/TVM
  \end{itemize}
}{}

\cventry{Fall 2018}{Part-Time Research Visitor}{Big Code Team}{Facebook}{Remote}{
  \begin{itemize}
    \item Applied deep learning to identify and highlight core code functionality
    in early ML4Code models.
  \end{itemize}
}{}


\cventry{Summer 2018}{Intern}{Software Engineering}{Facebook}{Boston}{
  \begin{itemize}
    \item Applied deep learning to code search and contributed to some of the
    earliest ML4Code models in this space.
  \end{itemize}
}{}

\cventry{Summer 2015}{Intern}{Data Science}{Cloudera}{San Francisco}{
  % \begin{itemize}
  %   \item Contributed multiple statistical tests and classical model implementations to a time series library for Spark (Github: \Colorhref{https://github.com/sryza/spark-timeseries}{Link})
  %   \item Contributed a distributed implementation of Kolmogorov-Smirnov test to Spark-MLlib (Github: \Colorhref{https://github.com/apache/spark/blob/master/mllib/src/main/scala/org/apache/spark/mllib/stat/test/KolmogorovSmirnovTest.scala}{Link})
  %   \item Wrote blog posts detailing technical contributions and use of time series library. (Blog: \Colorhref{http://blog.cloudera.com/blog/2015/10/continuous-distribution-goodness-of-fit-in-mllib-kolmogorov-smirnov-testing-in-apache-spark/}{Link})
  % \end{itemize}
}{}

\cventry{ 2011 -- 2014}{Full-Time Securitized Credit Research Associate}{Non-Agency Mortgages and US Housing}{Morgan Stanley}{New York}{
  % \begin{itemize}
  % \item Developed group analytics infrastructure to drive independence
  % from tools built/maintained by quant team
  % \item Learned q programming
  % language independently, quickly became productive in the language,
  % frequently helping others with technical q questions and eventually
  % helping in the review process of the latest \emph{Q for Mortals}
  % (Borror 2016) book
  % \item Introduced R development into the group and wrote base libraries for group
  % \item Led development of various research reports and investing themes
  % \end{itemize}
}{}

\cventry{Summer 2010}{Richard B. Fisher Scholar}{Fixed Income Generalist Sales and Fixed Income Credit Strategy}{Morgan Stanley}{New York}{}{}

\cventry{Summer 2009}{Douglas Paul Scholar}{Investment Banking and Alternative Investments}{Morgan Stanley}{New York}{}{}


% \section{Relevant Coursework}
% \vspace{5pt}

% \begin{itemize}
%     \item MIT: Computer Architecture, Theory of Computation, Database Systems, Machine Learning

%     \item NYU: Compiler Construction, Natural Language Processing, Speech Recognition, Programming Languages, Rigorous Software Development (an introduction to formal methods),
%     Principles of Software Security
% \end{itemize}


\section{Academic Work Experience}
\vspace{5pt}

\cventry{Fall and Spring 2021}{Advanced Undergraduate Research Class}{TA}{MIT}{}{}
\cventry{2015 -- 2016}{Graduate Course in Compiler Construction}{Grader}{NYU}{}{}
\cventry{Fall 2014}{Graduate Course in Programming Languages}{Teaching Assistant}{NYU}{}{}

\section{Language skills}

\vspace{6pt}

\begin{itemize}

\item \textbf{Programming Languages:} Proficient in: Python, Javascript/Typescript, R, C\#.

\vspace{6pt}

\item \textbf{Natural Languages:} Native fluency in English and Spanish. Working proficiency in German.
\vspace{6pt}

\end{itemize}

\section{Service}
\begin{itemize}
  \item \textbf{Program Committee ICSE 2024}
  \item \textbf{Program Committee Table Representation Learning Workshop (at NeurIPS) 2023}
  \item \textbf{Program Committee Table Representation Learning Workshop (at NeurIPS) 2022}
  \item \textbf{Artifact Evaluation Committee OOPSLA 2020}
  \item \textbf{Artifact Evaluation Committee CAV 2020}
  \item \textbf{Artifact Evaluation Committee PPoPP 2018}
  % \item \textbf{MIT PL Offsite 2017}:
  % I co-organized, with Ivan Kuraj, the MIT Programming Languages
  % offsite 2017. The event is meant to foster dialogue and ideas
  % among members of the MIT PL community and neighboring institutions.

  % \item \textbf{MIT Admitted Students' Visit Weekend Diversity Panel (2017, 2019, 2020)}:
  % I co-organized a diversity panel aimed
  % to provide a venue for prospective students to ask any questions
  % they might have about diversity at MIT and how we are working
  % towards improving our community.

  % \item \textbf{CSAIL Student Committee (2017 - Spring 2020)}:
  % I served as Treasurer on the CSAIL Student Committee. I managed the
  % group's budget and contributed to the organization of social
  % events, such as a weekly event featuring baked goods and socializing
  % among graduate students in CSAIL.
\end{itemize}

\section{Mentoring/Advising}
\begin{itemize}
  \item Jennifer McCleary (MIT) MEng Thesis: pancreatic cancer risk modeling (Fall 2019 - January 2020)
  \item Alex Berg (MIT) Undergraduate research: pancreatic cancer risk modeling (Summer 2020)
  \item Thomas Xiong (MIT) MEng Thesis: pancreatic cancer risk modeling (Fall 2020 - Spring 2021)
  \item Lori Zhang (MIT) Undergraduate research: pancreatic cancer risk modeling (Summer 2020 - Spring 2021)
  \item Harshit Joshi (Microsoft): PROSE Research fellow, automated program repair (Fall 2021 to July 2023 -- joining Stanford PhD program 2023)
  \item Mukul Singh (Microsoft): PROSE Research fellow, NL-to-Code (Spring 2022 to date)
  \item Abishai Ebenezer (Microsoft): PROSE Research fellow, automated program repair (Fall 2022 to July 2023)
  \item Jialu Zhang (Yale/Microsoft): Summer intern in the PROSE team, working on automated program repair (Summer 2022). Part of thesis committee.
\end{itemize}


\nocite{*}
\bibliographystyle{habbrvyr}
\bibliography{publications}



\end{document}


%% end of file `template.tex'.
